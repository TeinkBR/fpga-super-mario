\documentclass[a4paper,11pt]{article}
\usepackage[T1]{fontenc}
\usepackage[utf8]{inputenc}
\usepackage{lmodern}

\title{Super Mario in FPGA}
\author{Leticia L. Rodriguez}

\begin{document}

\maketitle
\tableofcontents
\newpage
\begin{abstract}
On September 13th, 1985 the Super Mario Bros. platforms video game was launched for Family Computers Famicom and Nintendo Entertainment System (NES). This project implements, on an FPGA board, the levels of the legendary game.
\end{abstract}

\section{Goal}
\section{Design}
\section{Implementation}
\subsection{Data preparation}
\subsection{Sprites}
\subsection{Maps}
\subsection{Code}
\subsubsection{MAP\_ADDRESS\_LOGIC}
\subsubsection{SPRITE\_ADDRESS\_LOGIC}
\subsubsection{PIXEL\_LOGIC}
\appendix
\section{Preparing the BlockRAM with images using GIMP}
\subsection{Image source}
\subsection{Step by step: JPG conversion to 256 colors binary}
\end{document}
